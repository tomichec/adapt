\documentclass[compress]{beamer}% [handout][compress]

% use handout option for one slide per page
% \mode<presentation>
%regular LaTeX title/author info
%\AtBeginSection[]{} % for optional outline or other recurrent slide

\setbeamertemplate{footline}[frame number]

\setbeamerfont{caption}{size=\tiny}
\beamertemplatenavigationsymbolsempty

\definecolor{darkgreen}{rgb}{0,0.7,0}

\mode<presentation>
{
\usetheme{Frankfurt} 
\usecolortheme{seahorse}
\setbeamercovered{transparent}
\setbeamercolor{math text}{fg=\notcolor}
\setbeamercolor{math text displayed}{fg=\notcolor}
}

\newenvironment{myalign}
{\align\color{\notcolor}}
{
  \nonumber
  \endalign
  \vspace{-1em}
}

\newenvironment{myitemize}
{\itemize\itemsep 0mm}
{\enditemize}

% \let\myalign\align
% \let\endmyalign\endalign
% \def\myalign{\begingroup \myalign}
% \def\endmyalign{\endmyalign \endgroup}

% \usepackage{psfrag}
% \psfrag{VFRAC}[c][c] {$\vfrac$}
\psfrag{STRESS}[c][c]{$\efstress$}
\psfrag{STRAIN}[c][c]{$\strain$}
\psfrag{PERMEAB}[c][c]{$\permeab$}
\psfrag{VISCOS}[c][c]{$\log_{10}(\viscos)$ (Pa$\cdot$s)}
\psfrag{VISC}[c][c]{$\log_{10}(\viscos)$}
\psfrag{VSCO}[c][c]{$\viscos$}
\psfrag{CURE}[c][c]{$\cure$}
\psfrag{TEMP}[c][c]{$\temp$ (\degree C)}
\psfrag{TMPR}[c][c]{$\temp$}
\psfrag{DCURE}[c][c]{$\cure_{,t}$ (1/s)}
\psfrag{TIME}[c][c]{$t$ (min)}

% \psfrag{CURE0}[l][c]{\hspace{-1.7em}\small$\cure=0.0$}
\psfrag{CURE1}[l][c]{\hspace{-1.5em}\small$\cure=0.02$}
\psfrag{CURE2}[l][c]{\hspace{-1.5em}\small$\cure=0.2$}
\psfrag{CURE3}[l][c]{\hspace{-1.5em}\small$\cure=0.4$}

\psfrag{TEMP1xxxx}[l][c]{\hspace{-2.2em}\small$\temp=100$\degree C}
\psfrag{TEMP2xxxx}[l][c]{\hspace{-2.2em}\small$\temp=130$\degree C}
\psfrag{TEMP3xxxx}[l][c]{\hspace{-2.2em}\small$\temp=180$\degree C}

% \input{./packages.tex} 
% \input{./macros_TS.tex}
\newcommand{\notcolor}{blue}
\newcommand{\+}[2]{\newcommand#1{{\color{\notcolor}#2}}}
\newcommand{\1}[2]{\newcommand{#1}[1]{{\color{\notcolor}#2}}}
\newcommand{\2}[2]{\newcommand{#1}[2]{{\color{\notcolor}#2}}}

\newcommand{\mat}{\boldsymbol }
\renewcommand{\exp}[1]{\mathrm{exp}\left({#1}\right)}

\+{\degree}{$^\circ$}
\+{\dd}{\mathrm{d}}
\1{\ddt}{\frac{\dd#1}{\dd t}}
\+{\pd}{\partial}

\+{\uuh}{u^h}
\+{\ggh}{g^h}
\+{\vvh}{v^h}
\+{\wwh}{w^h}
\+{\nel}{N}
\+{\young}{E}                   % young's modulus
\+{\area}{A}                    % cross-section area

\+{\shape}{\Phi}                % elementary shape (basis) function
\+{\stm}{\mat K}                % stiffness matrix
\1{\stme}{K_{#1}}       % stiffness matrix entry

\+{\estm}{\mat K^e}               % element stiffness matrix
\1{\estme}{\mat K^e_{#1}}       % element stiffness matrix entry

\+{\vdspl}{\vec d}
\+{\vfrc}{\vec F}

\1{\cO}{\mathcal{O}(#1)}

\+{\bcg}{\mathfrak{g}}
\+{\bch}{\mathfrak{h}}

\+{\press}{P}                   % pressure
\+{\pressatm}{P_\mathrm{atm}}   % atmospheric pressure
\+{\stress}{\sigma}             % total stress
\+{\efstress}{\bar\sigma}       % effective stress
\+{\strain}{\varepsilon}       % strain
\+{\rdense}{\rho_r}             % resin density

\+{\dx}{\Delta_x}                      % space step
\+{\dt}{\Delta_t}                      % time step
\1{\bigO}{\mathcal{O}(#1)}      % big O notation
\2{\disp}{u_{#1}^{#2}}          % shorthand for u
\2{\efstr}{\efstress_{#1}^{#2}}          % shorthand for u

\1{\ee}{\cdot 10^{#1}}          % times 10^x -- engineering notation

\+{\vfrac}{V_f}                 % fibre volume fraction
\+{\vzero}{V_0}                 % initial fibre volume fraction
\+{\vmax}{V_a}                  % maximal fibre volume fraction
\+{\permeab}{K}                 % permeability
\+{\kozeny}{k}                  % Kozeny constant
\+{\viscos}{\mu}                % viscosity
\+{\cure}{\alpha}               % cure
\+{\curegel}{\cure_g}          % cure gelation point
\+{\temp}{T}                    % temperature

\+{\viscosinf}{\mu_\infty} % theoretical viscosity at infinite temperature and no cure

\+{\spring}{A_s}                % spring constant

\+{\flow}{v}                    %flow through the fibre bed

\+{\duration}{D}                    % duration of the press

%%%%%%%%%%%%%%%%%%%%%%%%%%%%%%%%%%%%%%%%%%%%%%%%%%
% hyperlinks in the document
\usepackage{hyperref}
% https://tex.stackexchange.com/questions/823/remove-ugly-borders-around-clickable-cross-references-and-hyperlinks
\hypersetup{
    colorlinks,
    linkcolor={red!80!black},
    citecolor={blue!50!black},
    urlcolor={blue!80!black}
}
%%%%%%%%%%%%%%%%%%%%%%%%%%%%%%%%%%%%%%%%%%%%%%%%%%

%%%%%%%%%%%%%%%%%%%%%%%%%%%%%%%%%%%%%%%%%%%%%%%%%% 
% https://tex.stackexchange.com/questions/135649/make-citemy-reference-show-name-and-year
\usepackage{natbib}             % citations including name and year
%%%%%%%%%%%%%%%%%%%%%%%%%%%%%%%%%%%%%%%%%%%%%%%%%%


%%%%%%%%%%%%%%%%%%%%%%%%%%%%%%%%%%%%%%%%%%%%%%%%%%
% multicolumns in item lists (from viva slides)
\usepackage{multicol}

%%%%%%%%%%%%%%%%%%%%%%%%%%%%%%%%%%%%%%%%
% http://tex.stackexchange.com/questions/33969/changing-font-size-of-selected-slides-in-beamer
% \usepackage{lipsum} % lorem ipsum

% \newcommand\Fontvi{\footnotesize\selectfont}
%%%%%%%%%%%%%%%%%%%%%%%%%%%%%%%%%%%%%%%%

\institute{\footnotesize 
  College of Engineering, Mathematics and Physical Sciences  \\[1ex]
  \includegraphics[width=0.3\linewidth]{exeLogo.pdf}
}

\title{ADAPT Online Meeting:\\ Carbon Fibre Consolidation}

\author{ \textbf{\Large Tom\'a\v s Star\'y} \\ {\footnotesize\it advised by} \\[1ex] \textbf{\Large Tim Dodwell}
}


\date{
\it  Tue 29 November 2016
}

\newcommand{\highlight}[1]{\boxed{#1}}

\begin{document}
 
% title page
\frame{\titlepage}

% outline
\section[Outline]{}
\frame{\tableofcontents}


\section{Introduction}

\subsection{Problem Definition}

\begin{frame}[label=A]
  \frametitle{Physical Considerations}
  \begin{columns}
    \begin{column}{0.4\textwidth}
      \includegraphics{fibre_bed.pdf}
    \end{column}
    \begin{column}{0.6\textwidth}
      \begin{itemize}
      \item Controlled variables (in time $t$)
        \begin{itemize}
        \item $\temp$ -- temperature,
        \item $F$ -- applied force, or
        \item $u(L,t)$ -- boundary deformation.
        %   \begin{myalign}
        %   u(L,t) =
        %   \begin{cases}
        %     -Rt &\text{ for } t < \duration,\\
        %     -R\duration &\text{ for } t > \duration.
        %   \end{cases}
        % \end{myalign}
        \end{itemize}
        
      \item Resin flow through the pores
        \begin{myalign}
          \flow(x,t) = -\frac{\permeab(x)}{\viscos(t)} \press_{,x} 
        \end{myalign}
        \begin{itemize}
        \item $\permeab$ -- permeability,
        \item $\viscos$ -- resin viscosity,
        \item $\press$ -- resin pressure, $\press_{,x}= \dd P/\dd x$.
        \end{itemize}

      \item Stress in the fibre bed
        % 
        \begin{myalign}
          \stress(x,t) = \efstress(x,t) - \press(x,t)
        \end{myalign}
        % 
        \begin{itemize}
        \item $\efstress$ -- effective stress.
        \end{itemize}        
      \end{itemize}
    \end{column}
  \end{columns}
\end{frame}

\subsection{Simplified Model}

% \againframe{A}

\begin{frame}
  \frametitle{Fibre Bed Deformation}
  % \begin{columns}
  %   \begin{column}{0.6\textwidth}
      
      Stress in the fibre bed
      %
      \begin{myalign}
        \stress(x,t) = \efstress(x,t) - \press(x,t)
      \end{myalign}

      Assumptions:
      
      \begin{itemize}
      \item small displacement: $\efstress =E u_{,x}$
      \item static equilibrium: $\stress_{,x}= 0$
      \end{itemize}

      
      Using relation 
      $\press_{,x} =- \frac{\viscos(t)}{\permeab(x)} v$ and $\flow = -u_{,t}$
      we get
        \begin{myalign}
          E u_{,xx} - \frac{\viscos(t)}{\permeab(x)} u_{,t}  =0
        \end{myalign}
        \begin{itemize}
        \item $u_{,xx} =\dd^2 u/ \dd x^2$, 
         $u_{,t} =\dd u/ \dd t$.
        \end{itemize}

  %   \end{column}
  %   \begin{column}{0.4\linewidth}
  %     When ${\viscos(t)},{\permeab(x)},E=1$
  %     \includegraphics[width=1.0\linewidth]{deform.eps}\\
  %     \includegraphics[width=1.0\linewidth]{flow.eps}
  %     % \includegraphics{cure-temp-viscos.pdf}
  %   \end{column}
  % \end{columns}
\end{frame}

\section{Finite Difference}
\subsection{Solution}

\begin{frame}
  \frametitle{Solution}

  Simplified problem when ${\viscos(t)},{\permeab(x)}$ and $E=1$:
\begin{align}
  \label{eq:heat}
  &u_{,xx}(x,t) - u_{,t}(x,t) + f(x) 
  = 0,&& u(0,t) = 0, \\\nonumber
             &u(x,0) = 0, u(L,t) =
            \begin{cases}
              -Rt &\text{ for } t < \duration,\\
              -R\duration &\text{ for } t > \duration.
            \end{cases}
\end{align}

\begin{figure}
  \centering
  \includegraphics[width=1.0\linewidth]{heat.pdf}
  % \caption{Finite difference solution of equation
  %   \eqref{eq:heat}. Panel (a) shows the dependence of the deformation
  %   of the fibre bed on time at the boundaries $x=0$ (violet line) and
  %   $x=1$ (yellow line) and in the interior points $x=0.2$ (green
  %   line) and $x=0.5$ (blue line) (b) shows the dependence of the
  %   displacement on the possition in the frame of reference $x$ at the
  %   initial conditions (violet line), at $t=0.1$ (green line), at
  %   $t=0.2$ (blue line) and at the end of the simulation $t=0.4$
  %   (yellow line).  Panel (c) shows colour coded map (left bar) of the
  %   deformation dependent on the time (vertical axis) and the frame of
  %   reference (horizontal axis). The a gray scale map from white
  %   colour which corresponds to small deformation to black which
  %   corresponds to maximal deformation of $-0.02$ is complemented by a
  %   contour lines ad four different values as specified in the legend.}
  \label{fig:findiff-heat}
\end{figure}

\end{frame}


\subsection{Numerical Error and Stability}
\begin{frame}
\frametitle{Numerical Error and Stability}

Finite difference algorithm:
\begin{align}
  \disp{i}{j+1}  =   \disp{i}{j} + \dt\left(\frac{\disp{i+1}{j} - 2\disp{i}{j} + \disp{i-1}{j}}{\dx^2}   + f(x_i) \right) +
\bigO{\dt} + \bigO{\dx^2}. \nonumber
   \label{eq:findiff-alg}
\end{align}
%
Solution $\disp{i}{j}\approx u(x_i,t_j)$ with discretisation
$ x_i = i\dx$, $ t_j = j\dt$ for $j = 1, 2, \dots, N_t$ and
$i = 1, 2, \dots, N_x$.

Stable solution for $  \dt < \frac{\dx^2}{2\max(D)}$.

Error $\errU = \norm{\hat{u} - u}$.
%
\begin{figure}
  \centering
  \includegraphics[width=\linewidth]{error.pdf}
  % \caption{Analysis of the error of the solution. Panel (a) shows the
  %   boundary between stable and unstable region in the $\dx$--$\dt$
  %   plane according to \eqref{eq:stable-boundary} where $\max(D) =
  %   1$. Panel (b) and (c) shows the norm of the error in the
  %   computation of the heat equation in (b) dependent on $\dt$ at
  %   $\dx=0.1$, in (c) as dependent on $\dx$ at $\dt=1e-7$.}
  \label{fig:error-analysis}
\end{figure}

\end{frame}


\section{Resin Flow Model}



\subsection{Fibre Volume Fraction}

\begin{frame}
  \frametitle{Fibre Volume Fraction}

  \begin{columns}
    \begin{column}{0.6\linewidth}
      Fibre volume fraction (FVF)
        \begin{myalign} 
          \vfrac(\strain) = \frac{\vzero}{1+\strain(x,t)}
        \end{myalign}
        \begin{itemize}
        \item $\vzero=0.55$ -- initial FVF,
        \item $\strain(x,t) = u_{,x}$ -- strain.
        \end{itemize}
    \end{column}
    \begin{column}{0.4\linewidth}
      \includegraphics{vf-strain.pdf}
    \end{column}
  \end{columns}
\end{frame}

\subsection{Fibre Bed Permeability}

\begin{frame}
  \frametitle{Fibre Bed Permeability}
  % \begin{columns}
  %   \begin{column}{0.6\linewidth}
      The equation of the flow
      \begin{myalign}
        \flow(x,t) = -\frac{\permeab(x)}{\viscos(t)} \press_{,x} .
      \end{myalign}
      
      Permeability by \citet{Dave1987b}
      \begin{myalign}
        \permeab(\vfrac) = \frac{r_f^2}{4\kozeny} \frac{(1-\vfrac)^3}{\vfrac^2} 
      \end{myalign}
      \begin{itemize}
        \itemsep 0mm
      % \item $r_f, \kozeny$ -- constants,
      \item $r_f=4\ee{-3}$~mm -- radius of fibre,
      \item $\kozeny=0.2$~1/s -- ``Kozeny'' constant, 
      \item $\vfrac$ -- fibre volume fraction (FVF).
        % \begin{myalign}
        %   \vfrac(\strain) = \frac{\vzero}{1+\strain(x)}
        % \end{myalign}
        % \begin{itemize}
        % \item $\vzero$ -- initial FVF,
        % \item $\strain(x)$ -- strain.
        % \end{itemize}
      \end{itemize}
  %   \end{column}
  %   \begin{column}{0.4\linewidth}
  %     \includegraphics{permeab-vf.pdf}\\
  %     % \includegraphics{vf-strain.pdf}
  %   \end{column}
  % \end{columns}

      \begin{figure}
  \centering
  \includegraphics[width=0.66\linewidth]{permeab.pdf}
  % \caption{Permeability of the fibre bed during the consolidation.
  %   The figure (a) The combination of dependence of volume fraction on
  %   the strain as shown in \figref{fig:consolidation}(a), combined
  %     with the dependence of the permeability on volume fraction shown
  %     in this figure panel (a), gives the dependence of the
  %     permeability as a function of strain as shown in panel (b).}
  \label{fig:permeab}
\end{figure}


  
\end{frame}




\subsection{Resin Viscosity}

\begin{frame}
  \frametitle{Resin Viscosity}
  % \begin{columns}
  %   \begin{column}{0.6\linewidth}
      Viscosity by \citet{Kenny1992}
      \begin{myalign}
        \viscos(\temp,\cure) =
        \viscosinf \exp{\frac{E_\viscos }{R \temp}} \left(\frac{\curegel}{\curegel - \cure}\right)^{A+B\cure} 
      \end{myalign}
      {\begin{multicols}{2}
      \begin{itemize}
        % \itemsep 0mm
        \item $\viscosinf=3.45\ee{-10}$~Pa$\cdot$s, % -- viscosity asymptote
        \item $E_\viscos = 7.6536\ee{5}$~J/mol, % -- activation energy
        \item $R = 8.617$~eV/K --  gas constant,
        \item $\curegel = 0.47$ -- cure at  gelation,
        \item $A, B$ -- constants,
      % \item $\viscosinf,E_\viscos, R, A, B$ -- constants,
      % \item $\curegel = 0.47$ -- cure at gelation,
      \item $\cure$ -- degree of cure.
      \end{itemize}
    \end{multicols}}

      \begin{figure}
  \centering
  \includegraphics[width=\linewidth]{viscosity.pdf}
  % \caption{Viscosity of the resin. Panel (a) shows the dependence of the
  %   viscosity on the temperature for three different values of
  %   $\cure$: magenta line $\cure=0.1$, green line $\cure=0.2$, cyan
  %   line $\cure=0.4$, panel (b) shows the dependence of the viscosity
  %   on the degree of cure for three different temperatures: magenta
  %   line $\temp = 100$\degree C, green line $\temp = 130$\degree C, and cyan line
  %   $\temp = 180$\degree C.}
  \label{fig:viscos}
\end{figure}
  
\end{frame}


\subsection{Resin Cure}
\begin{frame}
  \frametitle{Resin Cure}
  % \begin{columns}
  %   \begin{column}{0.6\linewidth}
  Degree of cure
  \begin{myalign}
    \cure_{,t} = A_\cure \exp{-\frac{E_\cure }{R\temp}} \cure^m
    (1-\cure)^n 
  \end{myalign}
  \begin{itemize}
  \item $A_\cure=1.53\ee{5}$~1/s,
  \item $E_\cure=6.65\ee{4}$~J/mol,
  \item $m=0.813$, $n=2.74$.
    % \item $A_\cure, E_\cure, m, n$ constants.
  \end{itemize}

  \begin{figure}
    \centering
    \includegraphics[width=\linewidth]{dcure.pdf}
    % \caption{Rate of cure of the resin. Panel (a) shows the dependence of the
    % rate of cure on the temperature for three different values of
    % $\cure$: magenta line $\cure=0.1$, green line $\cure=0.2$, cyan
    % line $\cure=0.4$, panel (b) shows the dependence of the rate of cure 
    % on the degree of cure for three different temperatures: magenta
    % line $\temp = 100$\degree C, green line $\temp = 130$\degree C, and cyan line
    % $\temp = 180$\degree C.}
    \label{fig:cure}
  \end{figure}


  %   \end{column}
  %   \begin{column}{0.4\linewidth}
  %     % \includegraphics{viscosity-map.pdf}\\
  %     \includegraphics{dcure-map.pdf}
  %   \end{column}
  % \end{columns}
  
\end{frame}




\begin{frame}
  \frametitle{Resin Viscosity in Autoclave Processing}
  Autoclave temperature protocol
  \begin{itemize}
  \item first $30$ minutes: from $\temp_0=30$\degree C to $\temp_f=107$\degree C,
  \item next $30$ minutes to allow resin flow at $\temp_f$,
  \item next $30$~minutes: from $\temp_f$ to $\temp_c=176$\degree C,
  \item final $30$~minutes at $\temp_c$ to complete resin cure.
  \end{itemize}
\begin{figure}
  \centering
  \includegraphics[width=0.66\linewidth]{cure.pdf}
  % \caption{Time evolution of the cure of the resin in the autoclave
  %   experiment (a), and related evolution of the viscosity of the
  %   resin (b). The ODE \eqref{eq:cure} is subject to the autoclave
  %   temperature protocol (shown in green lines, labels on the right
  %   axis in (b) applies to both panels).}
  \label{fig:cure-exp}
\end{figure}

  % \centering
  % \begin{tabular}{l@{\hspace{3em}}l}
  % \includegraphics{cure-temp-cure.pdf}&
  % \includegraphics{cure-temp-viscos.pdf}
  % \end{tabular}
\end{frame}

\section{Effective Stress}

\begin{frame}
  \frametitle{Effective stress}

  % \begin{columns}
  %   \begin{column}{0.6\textwidth}
      Effective stress by \citet{Gutowski1987}
      \begin{myalign}
        \efstress(\vfrac) = \spring
        \frac{\frac{\vfrac}{\vzero} -1}{\left(\frac{1}{\vfrac} -\frac{1}{\vmax}\right)^4} 
      \end{myalign}
      \begin{itemize}
      \item $\spring$ -- ``spring constant'',
      \item $\vzero = 0.55$ -- initial FVF,
      \item $\vmax = 0.68$ -- maximal FVF.
      % \item Fibre volume fraction (FVF)
      %   \begin{myalign}
      %     \vfrac(\strain) = \frac{\vzero}{1+\strain(x)} 
      %   \end{myalign}
      %   \begin{itemize}
      %   \item $\strain(x)$ -- strain.
      %   \end{itemize}
      \end{itemize}
  %   \end{column}
  %   \begin{column}{0.4\textwidth}
  %     \includegraphics{stress-strain.pdf}
  %     % \includegraphics{vf-strain.pdf}
  %   \end{column}
  % \end{columns}

      \begin{figure}
  \centering
  \includegraphics[width=\linewidth]{consolidation.pdf}
  % \caption{Mechanical properties of the material during the consolidation of the carbon fibres. The combination of dependence of volume fraction on the strain as shown in panel (a), combined with the dependence of stress on volume fraction shown in panel (b), gives the dependence of the stress as a function of strain as shown in panel (c).}
  \label{fig:consolidation}
\end{figure}
\end{frame}

\section{Conclusions}

\begin{frame}
  \frametitle{Conclusions}
  Results
  \begin{itemize}
  \item finite difference model for resin flow,
  \item error and stability analysis,
  \item fibre cure model,
  \item description of relevant physics.
  \end{itemize}

  Further work
  \begin{itemize}
  \item fibre bed stiffening,
  \item finite element method,
  \item extending the code into 2D and 3D,
  \item implementation in DUNE.
  \end{itemize}
\end{frame}

% \againframe{A}

% \begin{frame}
  
%     Darcy's law for a flow through porous medium
%   \begin{myalign}
%     v(x,t) = -\frac{\permeab(\vfrac)}{\viscos(\temp,\cure)} \press_{,x}
%   \end{myalign}
%   \begin{itemize}
%   % \item $v$ -- velocity of the flow
%   \item $\press_{,x}$ -- gradient of resin pressure ($\dd \press/\dd x$)
%   \end{itemize}
  
% \end{frame}

\bibliographystyle{apalike}
\begin{frame}
  \frametitle{References}
  \bibliography{references}
\end{frame}


\end{document}
