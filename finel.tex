\documentclass[twoside,a4paper,12pt]{article}

\usepackage[T1]{fontenc}
\usepackage[utf8]{inputenc}
\usepackage{lmodern}
\usepackage{amsmath}
\usepackage{color}

\usepackage[left=3cm,right=3cm,top=2cm,bottom=2cm,asymmetric]{geometry}

\newcommand{\notcolor}{blue}
\newcommand{\note}[1]{{\color{red}\newline(#1)}}
\newcommand{\QM}{{\color{red}(?)}}

\newcommand{\+}[2]{\newcommand#1{{\color{\notcolor}#2}}}
\newcommand{\1}[2]{\newcommand{#1}[1]{{\color{\notcolor}#2}}}
\newcommand{\2}[2]{\newcommand{#1}[2]{{\color{\notcolor}#2}}}

\newcommand{\mat}{\boldsymbol }

\+{\dd}{\mathrm{d}}
\+{\uuh}{u^h}
\+{\ggh}{g^h}
\+{\vvh}{v^h}
\+{\wwh}{w^h}
\+{\nel}{N}
\+{\young}{E}                   % young's modulus
\+{\area}{A}                    % cross-section area

\+{\shape}{\Phi}                % elementary shape (basis) function
\+{\stm}{\mat K}                % stiffness matrix
\1{\stme}{\mat K_{#1}}       % stiffness matrix entry

\+{\estm}{\mat K^e}               % element stiffness matrix
\1{\estme}{\mat K^e_{#1}}       % element stiffness matrix entry

\+{\vdspl}{\vec d}
\+{\vfrc}{\vec F}

\begin{document}

\section{Statement of the system}

This program aims to use finite elements methods to solve the boundary
value problem

\begin{align}
  \young(x) \area(x) u_{,xx} + f(x) = 0, \label{eq:bvp}
  &&u(L) = g, u_{,x}(0) = -h,
\end{align}
where $L$ is the length, $\young(x)$ is Young's modulus and $\area$ is
the cross section area, $u(x)$ is a deformation of the of the object
in consideration, and shorthand notation for a derivative is written
after a coma in a subscript, e.g. $u_{,x} = \dd u /\dd x$.  The
function $f(x)$ represents the body forces. The $g$ and $h$ are
essential and natural boundary conditions respectively.

Let $u(x)$ be a solution of system \eqref{eq:bvp} then we may write
%
\begin{align}
  0 = \int_0^L (\young(x) \area(x) u_{,xx} + f(x)) w(x) \dd x 
\end{align}
and $w(x)$ is a chosen \emph{weighting (test) function} that satisfies a
criteria $w(L)=0$.
%
Using integration by parts we get
%
\begin{align}
  \int_0^L \young(x) \area(x) u_{,xx} w(x) \dd x 
  -\int_0^L f(x) w(x) \dd x  =
   \int_0^L u_{,x} (\young(x) \area(x) w(x))_{,x} \dd x -\nonumber \\ -
  (u_{,x} (\young(x) \area(x)  w(x))\big|_0^L -
  \int_0^L f(x) w(x) \dd x =  
  %
  \int_0^L u_{,x} (\young(x) \area(x) w(x))_{,x} \dd x -\nonumber \\ -
  (u_{,x}(L) (\young(L) \area(L)  w(L) -
  u_{,x}(0) \young(0) \area(0)  w(0)) 
  -\int_0^L f(x) w(x) \dd x = 0
\end{align}
%
and using the boundary values and the property of the weighting
function, the previous simplifies as
%
\begin{align}
  \int_0^L u_{,x} (\young(x) \area(x) w(x))_{,x} \dd x =  
  \int_0^L f(x) w(x) \dd x + h \young(0) \area(0)  w(0)
  \label{eq:weak-exact}
\end{align}
%
this method is called a \emph{weak (variational)}.

The solution of this system can be found by strong (classical) method
or weak (variational) method. The equivalence with the \emph{strong
  (classical)} method goes under a name \emph{fundamental lemma} which we will not show in here.

To solve the system numerically we will use Galerkin approximation
method. This method uses \eqref{eq:weak-exact} and provides an
approximation of the boundary value problem \eqref{eq:bvp} as
%
\begin{align}
a(\young\area\wwh,\uuh) = (\wwh,f) + h \young(0)\area(0)\wwh(0) \label{eq:weak-approx}
\end{align}
%
where we define the operators $a(y,z) = \int_0^L y_{,x} z_{,x} \dd x$
and $(y,z) = \int_0^L y z \dd x$ for arbitrary functions $y = y(x)$
and $z= z(x)$.

We construct the function
\begin{align}
  \uuh  = \vvh + \ggh \label{eq:galerkin-construc}
\end{align}
%
where $\ggh$ is given by satisfying the essential boundary condition
$\ggh(L) = g$ and $\vvh$ corresponds to the displacement. We
substitute \eqref{eq:galerkin-construc} into \eqref{eq:weak-approx} to
obtain
%
\begin{align}
  a(\young\area\wwh,\vvh) = (\wwh, f) + \young(0)\area(0)\wwh(0)h - 
  a(\young\area\wwh,\ggh)\label{eq:galerkin}
\end{align}
%
the weighting function can be defined as
%
\begin{align}
  \wwh = \sum_{i=0}^{\nel-1} c_i \shape_i \label{eq:wwh-sum}
\end{align}
%
where $\shape_i$ is so-called a \emph{basis (shape) function} for
element $i$. Similarly, we define
%
\begin{align}
  \vvh = \sum_{i=0}^{\nel-1} d_i \shape_i \label{eq:vvh-sum}
\end{align}
where $c_i, d_i$ are constants.

After substitution of \eqref{eq:wwh-sum} and \eqref{eq:vvh-sum} into
the \eqref{eq:galerkin} and using the bilinearity property of
operators $a(\cdot,\cdot)$ and $(\cdot,\cdot)$ we get an expression
%
\note{Is it for a single element? What about the integration limits
  and $c_i$?}
%
\begin{align}
  \sum_{j=0}^{\nel-1} a(\young\area\shape_i, \shape_j) d_j = (\shape_i, f) + \young(0)\area(0)\shape_i(0) h - a(\young\area\shape_i,\shape_\nel)g. \label{eq:galerkin-row}
\end{align}
%
Using a notation
%
\begin{align}
  \stme{ij} =& a(\young\area\shape_i,\shape_j) = \int_0^L \young(x)\area(x)\shape_{i,x}(x),\shape_{j,x}(x) \dd x \\
  F_i =& (\shape_i,f) + \young(0)\area(0)\shape_i(0) h - a(\young\area\shape_i,\shape_{\nel}) g .
\end{align}
Then equation \eqref{eq:galerkin-row} can be rewritten as
%
\begin{align}
  \sum_{j=0}^{\nel-1} \stme{ij} d_j = F_i
\end{align}
or more concisely as
\begin{align}
\stm \vdspl = \vfrc
\end{align}
where $\stm$ is a \emph{stiffness matrix}, $\vdspl$ is a
\emph{displacement vector} and $\vfrc$ is a \emph{force vector}.

\section{Problem Statement}

We define the basis function and find its derivative for each element
$i$ of $\nel$ elements. In the middle elements we get
\begin{align}
\shape_i(x) =& \frac{x-x_{i-1}}{h_{i-1}}, &\shape_{i,x}(x) =& \frac{1}{h_{i-1}},&&\text{for } x_{i-1} \leq x \leq x_i \\
\shape_i(x) =& \frac{x_{i+1}-x}{h_{i}},  &\shape_{i,x}(x) =& -\frac{1}{h_{i}}, &&\text{for }x_i \leq x \leq x_{i+1} \\
\shape_i(0) =& 0,                       &\shape_{i,x}(0) =& 0,                &&\text{elsewhere.}
\end{align}
%
whereas for the boundary node we have
%
\begin{align}
\shape_0(x) =& \frac{x_1-x}{h_{0}},             &\shape_0(x) =& -\frac{1}{h_{0}},                          &&\text{for } x_0 \leq x \leq x_{1} \\
\shape_\nel(x) =& \frac{x-x_{\nel-1}}{h_{\nel-1}},&\shape_\nel(x) =& \frac{1}{h_{\nel-1}},             &&\text{for } x_{\nel-1} \leq x \leq x_\nel .
\end{align}

Now we can write the stiffness matrix for a single element, i.e. in an
interval interval $[x_{i-1}, x_i]$ (where $i = 1, \hdots, \nel$ ) as
%
\begin{align}
\stme{ij} = a(\young\area\shape_i,\shape_j) = \int_{x_{i-1}}^{x_i}
\end{align}




\end{document}
